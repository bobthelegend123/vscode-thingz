\renewcommand{\familydefault}{\sfdefault}
\documentclass{scrartcl}
\usepackage{amsmath}
\usepackage[utf8]{inputenc}
\usepackage{natbib}
\usepackage{graphicx}

\begin{document}
\title{EuPhO Selection test 2022}
\author{Xin Rui}
\date{March 2022}
\setlength\parindent{0pt}

\maketitle

% \noindent\fbox{%
%     \parbox{\textwidth}{%
%         \textbf{Important:}We shall assume that the radius and shape of the rotor blades is kept constant(can vary that later) and that airflow is laminar.
%     }%
% }

\section{Projectile}

\begin{enumerate}

      \item A particle is launched as angle $\theta$ with respect to the horizontal at initial velocity $v_0$. Derive the  equation for its trajectory, $y(x)$.
      \item Calculate the horizontal range of the projectile, hence find the angle at which the range is maximised.
      \item Projetile evenlope question: given that the equation of the line is  in the form of $y=-b+ax^c$, find the value of $a$,$b$ and $c$.
      \item There is a wll with height $h$ in front of the particle. The wall has negligible thickness. Find the minimum velocity needed for the particle to fly over the wall.
      \item There is a cube with side length $L$ in front of the particle, sketch the optimal trajectory of the particle for it to fly over the cube with minimum velocity, $v_{min}$. Find $v_{min}$.
      \item Same question except replace the cube with a sphere with radius $R$ (sketch the optimal trajectory and minimum velocity).

\end{enumerate}

\section{Metal bawl}
A metal ball with mas $M$, radius $R$, specific heat capacity $c$, and linear thermal expansion coefficient $\alpha$. A certain amount of heat $Q$ was supplied to heat up the metal ball.
\begin{enumerate}
      \item Find the change in temperature of the metal ball when
            \begin{itemize}
                  \item the ball is placed on the table
                        $$Q=mc\Delta T + mgR\alpha\Delta T$$
                        as heat is required to both increase the balls temperature and increase its GPE.
                        Making $\Delta T$ the subject gives:
                        $$\Delta T=\frac{Q}{mc+mgR\alpha}$$
                  \item the ball is hung by a string
                        $$mc\Delta T= mgR\alpha\Delta T+Q$$
                        $$\Delta T = \frac{Q}{mc-mgR\alpha}$$
            \end{itemize}
      \item A heat engine is blah blah blah.... heat spplied then do work then some heat wasted blah blah. Now, imagine the following process
            \begin{enumerate}
                  \item the ball is first placed on the table. A certain amount of heat, $Q$ is supplied to heat up the ball from its initial temperature $T_0$ to $T_1$.
                  \item the table is then removed, and the ball (currently at temperature $T_1$) is now tied to a string and hung from the ceiling. $Q'$ is removed from the ball to cool its temperature down back to $T_0$.
                  \item The ball is placed back on the table
            \end{enumerate}
            Express $T_1$ and $Q'$ in terms of known constants as well as $T_0$ and $Q'$.
            To find $T_1$, idk just add $\Delta T_{gain}$ to $T_0$ lol not sure.

            $$T_1=T_0+\Delta T_{gain}=\frac{Q}{mc+mgR\alpha}$$

            $$mc\Delta T_{loss} =Q'+mgR\alpha\Delta T_{loss}$$
            $$T_{loss}=\frac{Q'}{mc-mgR\alpha}$$

            To find $Q'$, we know that $\Delta T_{gain}=\Delta T_{loss}$.

            $$\frac{Q'}{mc-mgR\alpha}=\frac{Q}{mc+mgR\alpha}$$
            $$Q'=Q \bigg(\frac{c-gR\alpha}{c+gR\alpha}\bigg)$$

      \item Find the change in height of the centre of mass of the steel ball after this cycle was repeated.

            I actually got 0 for this question :) it kinda makes sense since the initial and final temperature of the balls are the same so the radius shouldn't change. Right??
      \item Calculate the thermodynamic efficiency of this process.

            Idk what i am even doing at this point since my answer was 0 for the previous question. So i just listed down the equations
            $$\eta=\frac{mg\Delta h}{Q-Q'}=\frac{1}{2}$$
            but now when i think about it, i think the numerator should be $$2mg\Delta h$$, so the efficiency would be 1 instead!
      \item As $g$, the gravitational acceeration becomes large, the efficiency will exceed 100 percent, prove that from your previous calculation.
      \item Hence, account for the difference between the theoretical efficiency and the actual efficiency.
\end{enumerate}

\end{document}

