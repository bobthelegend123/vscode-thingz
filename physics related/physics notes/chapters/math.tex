\chapter{Math}
\section{Math techniques}
\subsection{Approximating sums}
In general, 
\begin{equation}
    \lim_{n\to \infty} \sum_{i=1}^N f(x_i) \Delta x= \int_a^b f(x) dx
\end{equation}
where $\Delta x=(b-a)/n$ and $x_i=a+i\Delta x$

\section{Single variable calculus}
\section{Multi-variable calculus}
\subsection{Why does gradient points in direction of greatest ascent}
\begin{mybox}{green}{Intuition - Explanation 1}
    Easy explanation (imo): "Gradient" is defined as the vector summation of
    \begin{equation}
        \nabla{f}=\frac{\partial f}{\partial x} \hat{i}+\frac{\partial f}{\partial y}\hat{j}+ .....
    \end{equation}
    the maximum rate of change along the individual axes. So naturally, the summation of these rate of changes would give the maximum rate of change as well.
\end{mybox}

\begin{mybox}{green}{Intuition/Proof - Explanation 2 (by \textbf{Yi Fan})}
    Single variable tangent line approximation:
    \begin{equation}
        f(x)\approx f(x_0)+ f'(x_0)(x-x_0)
    \end{equation}

    Multi-variable (2 in this case) tangent \textbf{plane} approximation:
    \begin{equation}
        f(x,y)-f(x_0,y_0) \approx f_x(x_0,y_0)(x-x_0) + f_y(x_0,y_0)(y-y_0)
    \end{equation}
    where $f_x$ and $f_y$ denotes partial derivatives with respect to $x$ and $y$ respectively.

    \begin{flushleft}
        Locally if we zoom in near $(x_0, y_0)$ this shows that any nice function is just a plane.
    \end{flushleft}
    \begin{flushleft}
        Then now suppose our plane is given by $z = ax + by$, where we take reference point to be the origin (otherwise we can just translate the function).
    \end{flushleft}
    \begin{flushleft}
        Now you can think of it as $z = (a,b)\cdot (x,y)$ where we have a dot product, and we want to find the direction $(x,y)$ that maximises this. ($a = f_x(x_0, y_0)$ and $b = f_y(x_0,y_0)$)
    \end{flushleft}
    \begin{flushleft}
        But dot product is maximised when the vector points in parallel direction. So we must have $(x,y) = k(a,b)$ for some constant $k$.
    \end{flushleft}
    \begin{flushleft}
        So essentially, we are trying to move $(x,y)$ around such that we can get the maximum $z$ value. And we realised that when $(x,y)$ is aligned with the gradient, the maximum $z$ value is achieved. Hence, \textbf{gradient always points in direction of greatest ascent.}
    \end{flushleft}

\end{mybox}

\subsection{Divergence}
\subsection{Curl}

\section{Differential equations}
\subsection{Linear differential equations - from Hexiang's slides}
If you want to solve something simple like
$$\frac{dy}{dx}=ay,$$
one could use separation of variables
$$\int dx=\frac{1}{a}\int \frac{1}{y} dy$$
$$x+d=\frac{\ln(y)}{a}+c$$
$$ax+c=\ln(y)$$
$$y=e^{ax+c}=e^{ax}e^c$$
which can be generalised to
$$y=Ae^{ax}.$$
Alternatively, we can also guess some solution, maybe $y=e^{cx}$ will do. $\frac{dy}{dx}=ce^cx$. Hence, $c=a$. This solution wil still hold true when we multiply it by an arbitrary constant ($y=Ae^{ax}$).

\subsection{Solution to $\ddot x=-\omega^2 x$}
Let's try subbing in $x=Ae^{ct}$ again and see what we get,
$$\ddot x= c^2 A e^{ct}$$
$$c^2 A e^{ct}=-\omega^2 Ae^{ct}$$
$$c=\pm i\omega$$
So there are 2 possible values for $c$, hence 2 solutions for $x$ ($Ae^{i\omega t}$ or $Ae^{-i\omega t}$)
For a linear differential equation (in this case it is), the most general solution would be the \textbf{linear combination} of the individual solutions
$$\boxed{x=A_1 e^{i\omega t} + A_2 e^{-i\omega t}}$$
Using \textbf{Euler's identity}
$$x=A_1(\cos(\omega t)+i \sin(\omega t))+A_2(\cos(-\omega t)+i \sin(-\omega t))$$
$$x=(A_1+A_2)\cos(\omega t)+i(A_1-A_2) \sin(\omega t)$$
$$x=B\cos(\omega t)+C\sin(\omega t)$$
which can then be converted to the form which we are more familiar with (r-formula): $$\boxed {x=A\sin(\omega t + \phi) \quad \textsf{or}\quad x=A\cos(\omega t + \phi).}$$ Since we absorbed $i$ into the constants, $A$ and $\phi$ are both complex right now. However, we can just impose the conditions that they must always be real for physical purposes.
\subsection{Integrating fractor}
If you have an equation in the form of

\begin{align}
    \frac{dy}{dx}+P(x)y &= Q(x)\\
    M(x)\frac{dy}{dx}+M(x)P(x)y &= M(x)Q(x)\\
\end{align}

You can multiply both side by an integrating factor $M(x)$, such that the LHS becomes the result of a product rule. This is useful because once we can get it to be like product rule, we can integrate both side and solve for $y$.

\begin{align}
    M(x)y &= \int Q(x)M(x)dx\\
    y &= \frac{\int Q(x)M(x)dx}{y}
\end{align}

How do we find this integrating factor?

\begin{equation}
    M'(x)=M(x)P(x)
\end{equation}

\begin{equation}
    \boxed{M(x)=\exp\bigg(\int P(x)dx \bigg)}
\end{equation}

\section{Cool integrals}
\subsection{Gaussian integral}
\begin{equation}
    \int_{-\infty}^\infty e^{-x^2}dx=\sqrt{\pi}
\end{equation}

\begin{equation}
    \int_{-\infty}^\infty e^{-\alpha x^2}dx=\sqrt{\frac{\pi}{\alpha}}
\end{equation}
Very useful in thermodynamics. 