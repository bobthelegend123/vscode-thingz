\chapter{Special Relativity}
There are 3 fundamental effects of special relativity
\begin{itemize}
    \item Loss of simultaneity
    \item Time dilation
    \item Length contraction
\end{itemize}
\section{kinematics}
Let the frame $S'$ be moving at speed $v$ relative to a stationary frame $S$. Denote all quantities (for e.g. time) in the $S'$ frame with a prime ($'$).  
\subsection{basics}
\textbf{Time dilation}
\begin{equation}
    t'=\frac{t}{\gamma}
\end{equation}
\textbf{Length contraction}
\begin{equation}
    \ell'=\frac{\ell}{\gamma}
\end{equation} \footnote{$t'$ and $\ell'$ are the time and length in the moving frame $S'$.}
where $\gamma=1/\sqrt{1-\frac{v^2}{c^2}}$
Since $c$ is the maximum possible speed a thing can reach, $\gamma$ will always be \textbf{greater than or equal to} 1. 
\subsection{Lorentz transformation}
When $S'$ is moving in the positive $x$ direction relative to $S$, the Lorentz transformation is written as follows: 

\begin{multicols}{2}
    \begin{equation}
        \begin{bmatrix}
            ct' \\
            x'
        \end{bmatrix}
        =
        \begin{bmatrix}
            \gamma & -\beta \gamma \\
            -\beta \gamma & \gamma 
        \end{bmatrix}
        \begin{bmatrix}
            ct\\
            x 
        \end{bmatrix}
    \end{equation}\break


    \begin{equation}
        \begin{bmatrix}
            ct \\
            x
        \end{bmatrix}
        =
        \begin{bmatrix}
            \gamma & \beta \gamma \\
            \beta \gamma & \gamma 
        \end{bmatrix}
        \begin{bmatrix}
            ct'\\
            x'
        \end{bmatrix}
    \end{equation}
\end{multicols}

This is lorentz transformation in the matrix form, where the transformation matrix is known as the \textbf{lorentz boost}. \textcolor{blue}{To boost from the $S$ frame to the $S'$ frame, we have the "$-$" sign. To boost from the $S'$ frame to the $S$ frame, we have the "$+$" sign.}


\begin{equation}
    \begin{cases}
      \Delta x= \gamma (\Delta x' + v \Delta t') \\
      \Delta t= \gamma (\Delta t'+v \frac{\Delta x'}{c^2})\\
    \end{cases}      
\end{equation} 

\begin{equation}
    \begin{cases}
      \Delta x'= \gamma (\Delta x - v \Delta t) \\
      \Delta t'= \gamma (\Delta t - v \frac{\Delta x}{c^2})\\
    \end{cases}      
\end{equation} 

To derive the equation for time dilation, we note that the person measuring it (either in $S$ or $S'$) has to be stationary (i.e. $v=0$).

From the equation above and setting $\Delta x'$ to 0, we see that $\Delta t=\gamma \Delta t'$. Since $\gamma\geq 1$, the time elapsed measured by the stationary observer in the $S'$ frame is always shorter than the time elapsed in $S$. 

However, if we set $\Delta x$ to 0 and use the second set of equation, we get $\Delta t'=\gamma \Delta t$. This seems to be a contradiction. But in fact, this scenario is very different as the stationary observer is now in the $S$ frame (and not the $S'$ frame).

\subsection{relativistic velocity addition}
Let $v_1$ be the speed of the object measured in the frame $S'$, which is moving at speed $v_2$ relative to the ground.

Writing out the lorentz transformation for $S'$, we get 

\begin{equation}
    \begin{cases}
        \Delta x= \gamma_2 (\Delta x'+v_2 \Delta t')\\
        \Delta t = \gamma_2 (\Delta t'+v_2 \frac{\Delta x'}{c^2})\\
    \end{cases}
\end{equation}
where $\gamma_2$ is the Lorentz factor associated with $v_2$.
As the speed measured in the lab frame ($S$) is defined as $\Delta x/\Delta t$, we get 
\begin{equation}
    \frac{\Delta x}{\Delta t}= \frac{\gamma_2 (\Delta x'+v_2 \Delta t')}{\gamma_2 (\Delta t'+v_2 \frac{\Delta x'}{c^2})} = \frac{v_1+v_2}{1+(v_1 v_2)/c^2}
\end{equation}
The derivation \textbf{above} only applies to velocity addition in the \textbf{longitudinal direction}. For \textbf{transverse velcoity addition}, we can use a similar method (Lorentz transformation: $\Delta y'=\Delta y= v \Delta t $)
\begin{equation}
    \frac{\Delta y}{\Delta t}=\frac{\Delta y'}{\gamma (\Delta t'+v \frac{\Delta x'}{c^2})} =\frac{\Delta y'/\Delta t'}{\gamma (1 +v \frac{\Delta x'/\Delta t'}{c^2})}= \frac{v_y'}{\gamma (1+v'_x\frac{v}{c^2})}
\end{equation}

Now, let's derive this result with \textbf{4-vectors}. The velocity 4-vector, $V^\mu$, is
\begin{equation}
    V^\mu = (\gamma_{v_1} c, \gamma_{v_1} \vec {v_1})
\end{equation}
Setting $c=1$ so that our equations do not get too messy, $V^\mu_{S'}=(\gamma_{v_1}, \gamma_{v_1}, 0,0)$ in $S'$. In $S$, $V^\mu_S = (\gamma_w, \gamma_w w, 0,0)$, where $w$ is the speed of the object in $S$ frame. Using the fact that \textbf{inner dot product of 4-vector is invariant across all reference frames}\footnote{In general, the inner dot product of 4 vectors, $K^\mu W_\mu=K_0W_0-K_1W_1-K_2W_2-K_3W_3$}, we can write
\begin{equation}
    V^\mu_{S'} \cdot V_{\mu,S'} = V^\mu_S \cdot V_{\mu,S}.
\end{equation}
Solving for $w$, we should get the same answer. 

\subsection{invariance and minkowski diagram}
Consider the quantity $(\Delta s)^2= c^2(\Delta t)^2- (\Delta x)^2$, where $s$ (dropping the $\Delta$ from now on) is known as the \textbf{invariant interval} because this quantity is invariant to coordinates (i.e. $ct-x=ct'-x'$, $(ct, x)$ is usually known as the \textbf{spacetime interval}). 

\begin{figure}[H]
    \centering
    \includegraphics[width=0.3\textwidth]{minkowski.png}
\end{figure}
This could be derived with \textbf{lorentz transformation} with $\beta =v/c$. 1 unit on the $ct$ axis would correspond to $\gamma\sqrt{(1+\beta^2)}$ on the $ct'$ axis. Hence,
\begin{equation}
    \frac{\textsf{one} \ ct' \ \textsf{unit}}{\textsf{one} \ ct \ \textsf{unit}}=\frac{\sqrt{1+\beta^2}}{\sqrt{1-\beta^2}}
\end{equation}

Simultaenous measurements can be represented by drawing a line perpendicular to the $ct$/$ct'$ axes and seeing the points of intersection of the line with the $ct$/$ct'$ axes. 

\section{dynamics}
\subsection{Energy and momentum}
\textbf{Energy}
\begin{equation}
    E=\gamma m c^2
\end{equation}
\textbf{Momentum}
\begin{equation}
    \mathbf{p}=\gamma m \mathbf{v}
\end{equation}
To justify these expressions, I think it is quite elegant to perform Taylor series expansion in the limit of $v \ll c$ for $1/\sqrt{1-x^2}$ \footnote{$(1+x)^{-\frac{1}{2}}=1-\frac{1}{2}x+\frac{3}{8}x^2-\frac{5}{16}x^3+\frac{35}{128}x^4...$}, you obtain

\begin{equation}
    \begin{cases}
        E=mc^2(1+\frac{1}{2}(\frac{v}{c})^2+\frac{3}{8}(\frac{v}{c})^3+\frac{5}{16}(\frac{v}{c})^4+...)=mc^2+\frac{1}{2}mv^2+...\\
        \mathbf{p}=m \mathbf{v}(1+\frac{1}{2}(\frac{v}{c})^2+\frac{3}{8}(\frac{v}{c})^3+\frac{5}{16}(\frac{v}{c})^4+...)=m\mathbf{v}+...\\
    \end{cases}
\end{equation}

There is this very important equation
\begin{equation}
    E^2=p^2c^2+m^2c^4
\end{equation}
Another pretty useful equation is from the fundamental equations describing relativistic momentum and energy, 
\begin{equation}
    \frac{\mathbf{p}}{E}=\frac{\mathbf{v}}{c^2}
\end{equation}

Moreover, the equations to convert energy and momentum between frames are also quite elegant. Let the particle be travelling at speed $v$ in frame $S'$, which is moving at speed $u$ relative to lab frame $S$, then the energy and momentum in the 2 frames are related through
\begin{equation}
    \begin{cases}
        E=\gamma_u(E'+vp')\\
        p=\gamma_u(p'+vE')\\
    \end{cases}
\end{equation} \footnote{$E'=\gamma_v m$ and $p'=\gamma_v m v$, note that it is $\gamma_v$ and not $\gamma_u$}

Through this equation, we note that $E$ and $p$ behave exactly the same way as $ct$ and $x$. Hence, we obtain the following equation
\begin{equation}
    m^2=E^2-p^2=E'^2-p'^2
\end{equation}
where the mass of the particle, $m$, is the \textbf{invariant quantity}.

\subsection{Relativistic Collisions}
It is important to note that this $E=\gamma mc^2$ here is not equivalent to kinetic energy($E-mc^2$). During collision, it is the \textbf{total energy} that is conserved, so always use $E$ and not KE. 

To solve collisions problems, it is nice to define a \textbf{4- momentum}, or $P=(E,p_x,p_y,p_z)$ \footnote{Note that it shld be $E/c$ to make it dimensionally consistent but since $c=1$, we can just write it like this}. The inner product of this "vector" is defined to be $a_1b_1-a_2b_2-a_3b_3-a_4b_4$. It is defined this way so that we can concisely write $m^2=E^2-p^2$ as 
\begin{equation}
    P\cdot P=m^2
\end{equation}

\subsection{Optimal Collisions}
The minimum energy configuration of a system of particles with fixed total momentum is the one where they all move with the same velocity. This is easiest to show by boosting to the center of mass frame (i.e. the frame with zero total momentum) and then boosting back.


\subsection{Force}






\section{4-vector}
The most basic 4-vector is the \textbf{4-displacement} between two points in spacetime, which we call events. The 4-displacement is a fundamentally invariant geoemtric object, which you can think of as an arrow pointing between these 2 events. 

Consider two inertial frame $S$ and $S'$, such that $S'$ is moving at speed $v$ in the positive x-direction relative to $S$. Their coordinates are calibrated such that their origins coincide at $t=t'=0$. The frames are said to be in \textbf{standard configuration}, and their change-of-coordinates transformation can be summarised by the following
\begin{equation}
    \begin{bmatrix}
        ct' \\
        x' \\ 
        y' \\ 
        z'
    \end{bmatrix}
    =
    \begin{bmatrix}
        \gamma & \beta \gamma &  & \\
        -\beta \gamma & \gamma &  & \\ 
         &  & 1 & \\ 
         &  &  & 1
    \end{bmatrix}
    \begin{bmatrix}
        ct\\
        x \\ 
        y \\ 
        z
    \end{bmatrix}
\end{equation}


