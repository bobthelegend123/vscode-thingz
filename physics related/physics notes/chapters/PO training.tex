\chapter{PO training}

\section{18/08/22}
\subsection{Gallilean Invariance}
All things in newtonian physics has to obey \textbf{gallilean invariance}, meaning that the laws have to hold true in every inertial reference frame (i.e. stationary frame/frame moving at constant $v$)

In school, we are taught that since $\mathbf{F}=\frac{d\mathbf{p}}{dt}$ where $\mathbf{p}=m \mathbf{v}$, by chain rule, we will get 

\begin{equation}
    \mathbf{F}=\frac{d (m\mathbf{\vec{v}})}{dt}=m\frac{d\mathbf{v}}{dt}+\mathbf{v} \frac{dm}{dt}
\end{equation}

Now, we shall verify the galliean invariance of the equation above. Let's have a frame $S'$ moving at constant velocity $\mathbf{u}$. The force, mass,velocity and time in $S'$ are denoted with a prime ($'$). Note that this frame is \textbf{inertial} and \textbf{all newtonian laws should hold in inertial frames}.

\begin{align}
    \mathbf{F}'=m'\frac{d\mathbf{v'}}{dt'}+\mathbf{v'} \frac{dm'}{dt'}
\end{align}

Since mass time and force are invariant, and $\mathbf{v'}=\mathbf{v}-\mathbf{u}$, we can rewrite it as 
\begin{equation}
    \mathbf{F}=m\frac{d\mathbf{v}}{dt}+(\mathbf{v}-\mathbf{u})\frac{dm}{dt}
\end{equation}
which is definitely not the same as the equation above. Hence, it is proven that the equation 7.1 is \textbf{not} gallilean invariant.

\subsection{rocket momentum}
However, in some special cases, the "wrong" equation can still be applied. 
Let the mass and velocity of a rocket be $m(t)$ and $v(t)$ respectively. 
At time $t'=t+\Delta t$
\begin{align}
    m(t)v(t) 
    &= m(t')v(t')+ (m(t')-m(t))(v(t')-u)\\
    &= (m(t)+\Delta m)(v(t)+\Delta v)+(\Delta m)(v(t)+\Delta v - u)
\end{align}

where $u$ is the speed of the fuel (particle) emitted.

Expanding the equation, we get
\begin{equation}
    m(t)v(t)=m(t)v(t)+m(t)\Delta v+\Delta m v(t)+\Delta m \Delta v -\Delta m v(t)-\Delta m \Delta v - \Delta m u 
\end{equation}

Simplifying everything, we get 
\begin{equation}
    m dv+u dm = 0
\end{equation}

\section{25/08/22}
\subsection{linear independence of solution}
From wikipedia: A sequence of vectors $\mathbf{v_1},\mathbf{v_2},...\mathbf{v_k}$ from a vector space $V$ is said to be \textbf{linearly dependent} if there exist scalars $a_1,a_2,...,a_k$, not all zero, such that 
\begin{equation}
    a_1 \mathbf{v_1}+a_2 \mathbf{v_2}+...+a_k \mathbf{v_k}= \mathbf{0}
\end{equation}
where $\mathbf{0}$ denotes the zero vector. 
basically, as long as the vectors are not parallel, they are \textbf{linearly independent}

\section{27/08/22}
\subsection{Newton-Raphson method}
for example if you wanna solve
\begin{equation}
    y=f(x)=0
\end{equation}
you can kinda use
\begin{equation}
    x_{n+1}=x_n-\frac{f(x_n)}{f'(x_n)}
\end{equation}

 but its kinda useless

 \subsection{PS 2 Q 7}
 how to decide the direction of static friction? do torque about the contact point (where static friction acts), and realise that torque by $F$ has to result in a clockwise rotation (into the page). This means that torque about the COM also has to be clockwise, meaning that static frction points in the opposite direction of $F$. 

\textbf{Kenneth Hong's solution}
Translational motion of the center of mass
\begin{equation}
    F \cos \theta - f_s= ma_{CM,x}
\end{equation}
Rotational mostion about the centre of mass $I=M(a^2+b^2)/2$
\begin{equation}
    f_s a - Fb=I\alpha
\end{equation}

with the additional no slip condition ($a_{CM,x}=a\alpha$) and solving for pure rolling, you get
\begin{equation}
    \begin{cases}
        a_{CM,x}=\frac{F/M}{1+I/Ma^2}(\cos\theta-\frac{b}{a})\\
        f_s=\frac{F}{1+I/Ma^2}(\frac{I}{Ma^2}\cos\theta+\frac{b}{a})
      \end{cases}
\end{equation}

Maximum tension:
\begin{equation}
    f_s \leq \mu_s (Mg-F \sin\theta)
\end{equation}
\begin{equation}
    T_{max}=\frac{\mu_s (1+I/Ma^2)/Mg}{(I/Ma^2)\cos\theta+\mu_s(1+I/Ma^2)\sin\theta+b/a}
\end{equation}

\subsection{PS 2 Q 8}

\indent Workdone by friction on a rolling and \textbf{slipping} object is not just $W=f_k \cdot d$, where $d$ is the displacement of the centre of mass.

It is actually
\begin{align}
    W_{f_k} 
    &= W_{\textsf{trans}}+W_{\textsf{rot}}\\
    &= \int_0^{\Delta x} f_k dx - \int_0^{\Delta \theta} \tau d \theta 
\end{align}

Note that it is a minus sign because in the question, friction is adding energy in the translational motion, but removing energy in the rotational motion. 

\section{08/09/2022}
\subsection{PS2 Q15}

\textbf{How to approximate} Let $\theta \to \epsilon \theta $, $\dot{\theta} \to \epsilon \dot{\theta}$ and $\ddot{\theta} \to \epsilon \ddot{\theta}$. We can let $\theta^2$ and $\dot{\theta}^2$ be 0 because at small angle, $\theta^2 \ll \theta$,$\dot{\theta}^2 \ll \dot{\theta}$. Simple harmonic oscillators are only to first order. ($\epsilon \ll 1$)

\subsection{symmetry of field}
\textbf{Only when these symmetries are satisfied, then one can use gauss's law to solve the problem}
\subsubsection{spherical}
About the center point, to prove that it is symmetric, the field has to remain the same after

\begin{compactitem}
    \item rotation about any axis through the point
    \item plane mirror through the point
\end{compactitem}

examples include electric field and gravitational field.

\subsubsection{cylindrical}
About the line/axis, the field has to remain the same after
\begin{compactitem}
    \item rotation
    \item translation
    \item plane
\end{compactitem}

\subsection{signs}
Since potential is defined as 
\begin{equation}
    \phi(a)=-\int_{\infty}^a \vec{g} \cdot \vec{dr}
\end{equation}

where $\vec{g}= -(GM/r^2)\hat{r}$ and $\vec{dr}=-r \hat{r}$, which results in 
$$\phi(a)=-\int_{\infty}^a \frac{GM}{r^2}dr \neq -\int_{\infty}^a -\frac{GM}{r^2}dr$$
where the second integral is what we use most of the time. \textbf{So why is it like this?}

This is because of the fact that you are integrating from $-\infty$ to $a$, which already accounted for the direction of $dr$, meaning that instead of $\vec{dr}=r \hat{r}\neq =-r \hat{r}$. 


\section{06/10/2022}
\subsection{Integration with substitution}
When doing substitution, ignore limit first. After integrating, convert new variable back to the old variable before substituting in the original limit. 

\subsection{Optics}
When using the thin lens equation ($1/u+1/v=1/f$)

\begin{enumerate}
    \item $f>0$ means mirror is converging, $f<0$ means mirror is diverging. 
    \item when $u,v>0$, image is real. When $u,v<0$, image is virtual.
\end{enumerate}

If two thin lens with focal length $f_1$ and $f_2$ stick together, the effective focal length is 
\begin{equation}
    \frac{1}{f_\textsf{eff}}=\frac{1}{f_1}+\frac{1}{f_2}
\end{equation}

\subsection{Thin film interference}
\begin{enumerate}
    \item For light travelling from $n_1$ to $n_2$, if $n_1>n_2$, there is no phase change upon reflection. If $n_1<n_2$ where is a phase change of 180 degrees upon reflection. (See Fresnel's equations).
    \item  $\lambda$ used in calculation should be the wavelength in vacuum. 
    \item Do all calculations with \textbf{optical distances}, which is the real measured distance, $d$, multiplied by the refractive index of the medium $n$. 
\end{enumerate}


\subsubsection{Trivial}
For composite lens/mirror problems, determine \textbf{number of imaging process.}