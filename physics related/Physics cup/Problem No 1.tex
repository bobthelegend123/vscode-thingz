\documentclass{article}
\usepackage[utf8]{inputenc}
\renewcommand{\familydefault}{\sfdefault}
\usepackage{amsmath}
\usepackage{natbib}
\usepackage{graphicx}
\usepackage[margin=1.2in]{geometry}
\usepackage{wrapfig}
\usepackage{float}
\usepackage{natbib}
\usepackage{graphicx}

\title{Problem 1 - Magnetized ball}
\date{13/12/2021}

\begin{document}
\maketitle

Since magnetisation is uniform, the bound current density within the sphere is:
\begin{equation}
    J=\nabla \times M=0
\end{equation}

On the other hand, the bound surface current density is 
\begin{equation}
    K=M \cdot n
\end{equation}
where $n$ is the surface normal vector.
We can find the vector potential of the uniformly magnetised sphere:
\begin{equation}
    A= \frac{\mu_0}{4\pi} \int \frac{K}{r} R^2 \sin\theta d\theta d\phi
\end{equation}
And find the B field using $B=\nabla \times A$

Finally, we obtain the magnetic field inside a uniformly magnetised sphere as
\begin{equation}
    B=\frac{2}{3} \mu_0 M
\end{equation}
The total magnetic dipole of this sphere would then be:
\begin{equation}
    m=M \frac{4}{3} \pi R^3
\end{equation}

\begin{equation}
    m= \frac{2 B \pi R^3}{\mu_0}
\end{equation}

The initial and final magnetic field inside the sphere can be found based on the graph given (the y intercepts):

\begin{equation}
    \Delta m= \frac{2 (B_{20}-B_{120}) \pi R^3}{\mu_0}= \boxed{0.85 A m^2} (to 2 s.f) 
\end{equation}

($B_{20}-B_{120}\approx 0.17 T$)


\end{document}